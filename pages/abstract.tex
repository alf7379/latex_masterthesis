\chapter{\abstractname}

%TODO: Abstract

% Importance of interactive semantic segmentation
Due to the rapid development of ML, semantic segmentation has become an important instrument in image processing.
Recently published methods also propose interactive approaches for the task of semantic segmentation \cite{Xu16-InteractiveObjectSelection} \cite{JG18-ClickCarving} \cite{Liew17-RegionalInteractiveImageSeg}.
Interactive methods require a form of user interaction, that provide additional information on the object of interest.
The user interaction is often realized by mouse clicks on the image and are further processed by a \gls{ml} based segmentation model \cite{Man18-DEXTR} \cite{Zha20-IOG}. 
Thereby, the task is transformed into a class-agnostic segmentation of a foreground object from the background.
Interactive methods can be used to support the label process in order to create new annotations for images.
These annotations may further be used as \gls{gt}, which opens the possibility to use these images with the new annotations for \gls{ml} methods \cite{Man18-DEXTR}.
This use case is of particular interest for the industrial environment due to two major factors.
First, due to increasing automation, modern image processing methods based on DL are increasingly applied.
Second, suitable data is very rare, which has the characteristics of individual industrial environments.
Interactive methods simplify the label process and therefore lower the inhibition threshold for creating new labels.

% Quick summary of the Chapters.
This thesis examines the abilities of multiple interactive segmentation methods by the performance of a benchmark study on real users..
In Chapter \ref{ord:ch2} the fields of semantic segmentation and interactive segmentation is introduced.
In the benchmark study four methods (polygon drawing, watershed transformation, \gls{dextr}\cite{Man18-DEXTR}, and \gls{iog}\cite{Zha20-IOG}) are used, described in Chapter \ref{ord:ch3}.
The setup of the benchmark study is presented in Chapter \ref{ord:ch4}.
In Chapter \ref{ord:ch5} the results of the benchmark study are evaluated and the methods are further examined by the use of statistical analysis.
% Research question(s)
In general it is examined if interactive segmentation methods based on \gls{ml} improve the labeling process.
Thereby, special attention is paid to the generalization capabilities over various domains and users.
Last, a conclusion summarizes the outcomes and gives an outlook on future work in Chapter \ref{ord:ch6}.


