\chapter{\abstractname}

% Importance of interactive semantic segmentation
Due to the rapid development of ML, semantic segmentation has become an important instrument in image processing.
Recently published methods also propose interactive approaches for the task of semantic segmentation \cite{Xu16-InteractiveObjectSelection} \cite{JG18-ClickCarving} \cite{Liew17-RegionalInteractiveImageSeg}.
Interactive methods require a form of user interaction, that provide additional information on the object of interest.
The user interaction is often realized by mouse clicks on the image and serves as additional guidance for the based segmentation model \cite{Man18-DEXTR} \cite{Zha20-IOG}. 
Thereby, the segmentation task is transformed into a class-agnostic segmentation of a foreground object from the background.
Interactive methods can be used to support the label process and therefore reduce the label costs.
This is particularly interesting as new datasets can be created more easily, allowing labels to be created for image domains, that are rather underrepresented in the research community.

% Quick summary of the Chapters.
This thesis examines the abilities of multiple interactive segmentation methods by the performance of a benchmark study on real users.
Thereby, the fields of semantic segmentation and interactive segmentation are introduced.
Four interactive segmentation methods (polygon drawing, watershed transformation, \gls{dextr}\cite{Man18-DEXTR}, and \gls{iog}\cite{Zha20-IOG}) are implemented and applied in a benchmark study.
In this benchmark study, these methods are used by real users to create annotation for different images.
The results of the benchmark study are evaluated by the use of statistical analysis.
Thereby, special attention is paid to the generalization capabilities over various domains and users.
% Research question(s)
In general it is examined to what extent interactive segmentation methods based on \gls{ml} improve the labeling process compared to rather classical methods without \gls{dl}.
It is also shortly evaluated if annotations, created by interactive segmentation method, are suitable as training data.
