% !TeX root = ../../main.tex
% Add the above to each chapter to make compiling the PDF easier in some editors.

\section{Summary}

%%% Review of contests covered in the thesis.

% Kreis schließen zur Einleitung

% Description of the single chapters (middle detail with focus on the outcome -> repeat the dicovered results)
% Chapter 1
Chapter \ref{ord:ch1} presents an exemplary use case for interactive segmentation methods to highlighted their relevance.

%%% Chapter 2 
In Chapter \ref{ord:ch2} the theoretical basics for the presented methods are explained.
Here \gls{ml} is introduced quickly, while the task of semantic segmentation is examined in detail.
Further, interactive segmentation and different approaches how to realize it are presented .
Of particular focus are the methods based on \gls{dl}, where the user sets clicks as a form of interaction.
Here is especially interesting how the user clicks are transformed into heatmaps and concatenated with the \gls{rgb} image to form a multidimensional model input as in the \gls{dextr} and \gls{iog} method.
Last, methods for statistical analysis are introduced to provide the basis for the evaluation.

%%% Chapter 3
Since in this thesis a benchmark study with users is performed, in Chapter \ref{ord:ch3} the individual methods of the benchmark study are presented.
As non-interactive baseline method polygon drawing is introduced, while as classical interactive segmentation method the watershed transformation is presented.
Furthermore, the \gls{dextr} \cite{Man18-DEXTR} and \gls{iog} \cite{Zha20-IOG} method are introduced as representatives of state-of-the-art interactive segmentation methods based on \gls{dl}.

%%% Chapter 4
In Chapter \ref{ord:ch4} the setup of the benchmark study is explained, which is the core component of this thesis.
The compilation of the image dataset for the benchmark study is described and which image attributes and domains were considered.
Further, is described what user statistics are recorded in the benchmark study and what the process of participating in the benchmark study was like.

%%% Chapter 5
Last, in Chapter \ref{ord:ch5} the results of the benchmark study and further experiments are evaluated.
To begin the research question posed in Chapter \ref{ord:ch1} is examined, if interactive segmentation methods based on \gls{dl} improve the labeling process compared to methods without \gls{dl}.
The comparison is based on the benchmark results and takes the accuracy and the annotation time into account as presented in Section \ref{ord:ch5:sec1}.
The accuracy of the \gls{dextr} method performed best, while the the \gls{iog} method performs worst, as presented.
Therefore, in this regard neither classical or \gls{dl} based methods are advantageous, but as single method \gls{dextr} is superior.
With respect to annotation time, \gls{dl} based methods are significantly faster than classical methods.

Further, the generalization capabilities over several domains are evaluated for the \gls{dextr} and \gls{iog} method in Section \ref{ord:ch5:sec2_generalization_domains}.
% TODO describe this.

In Section \ref{ord:ch5:sec_3_generalization_user}, the generalization over several users is evaluated.
% TODO describe this.

Further on, in Section \ref{ord:ch5:sec4_survey} the survey filled out by the benchmark participants and their results is presented.
The outcome of this small survey indicates, that a pleasant user experience requires a good performing method and vice versa.

Last, in Section \ref{ord:ch5:sec5_retrain} the statement from Manisis \etal is examined, that \gls{dl} models trained on \gls{dextr} annotations perform equally as models trained with the original \gls{gt} \cite{Man18-DEXTR}.
The experiments of this thesis show, that this statement can be confirmed in a limited way.
The different models perform similar based on the \gls{map}, but the models trained on the \gls{dextr} annotations perform significantly worse for high \gls{iou} values. 


% Void Pixel 
Since the PASCAL \gls{voc} dataset \cite{Eve20-PascalVOC} is an important and much used dataset in this field of study, it was also used in this thesis.
However, the characteristics of this dataset with special reference to the use of \gls{vp} should be presented more transparently.
As presented in Table \ref{tab:ch5:tests_on_datasets} the \gls{miou} differs significantly with and without the use of \gls{vp}.
By presenting the results with \gls{vp}, unrealistically high expectations are created that cannot be met on real world data or datasets without \gls{vp}.

% - Eval output

%%%
% Antwort auf Research Frage -> if interactive segmentation methods based on \gls{dl} improve the process of labeling based on the annotation time and accuracy.
%These classical methods may perform very well on certain images, but tend to reach their limitations as they deal with more complex structures. % -> this should be part of the evaluation.
% This is due to their rather simple processing of superficial characteristics \eg edges, textures, contrast and color.
% On the contrary \gls{dl} based methods are capable to examine images on a deeper level and so understand more complicated structures.

%%%
% Generalization over domains

% Generalization over domains

% Importance of 

%%% Future Work
 