% !TeX root = ../../main.tex
% Add the above to each chapter to make compiling the PDF easier in some editors.

\section{Summary}

In this thesis four interactive segmentation methods are evaluated by the use of a benchmark study performed by real users.
Thereby, the segmentation methods are rated based on their performance and generalization capabilities over various image domains and user.
%%% Review of contests covered in the thesis.

% Kreis schließen zur Einleitung

% Description of the single chapters (middle detail with focus on the outcome -> repeat the dicovered results)
% Chapter 1
Chapter \ref{ord:ch1} provides an introduction by highlighting the relevance of interactive segmentation methods and giving an overview about the structure of this thesis.

%%% Chapter 2 
In Chapter \ref{ord:ch2} the theoretical basics for this thesis are presented.
Here \gls{ml} is introduced quickly, while the task of semantic segmentation is examined in detail.
Further, interactive segmentation and different approaches are examined.
Of particular focus are the methods based on \gls{dl}, where the user interaction is realized by mouse clicks on the object to segment.
Here is especially interesting how the user clicks are transformed into heatmaps and processed in the segmentation model as in the \gls{dextr} and \gls{iog} method.
Last, methods for statistical analysis are introduced in this chapter to provide the basis for the evaluation.

%%% Chapter 3
The individual the methods, that are part of the benchmark study, are presented in Chapter \ref{ord:ch3}.
As non-interactive baseline polygon drawing is introduced, while as classical interactive segmentation method the watershed transformation is presented.
Furthermore, the \gls{dextr} \cite{Man18-DEXTR} and \gls{iog} \cite{Zha20-IOG} method are presented as representatives of state-of-the-art interactive segmentation methods based on \gls{dl}.

%%% Chapter 4
In Chapter \ref{ord:ch4} the setup of the benchmark study is explained, which is the core component of this thesis.
The compilation of the image dataset for the benchmark study is described by the presentation of the considered image attributes and the image domains $ Standard $, $ Urban $, $ Industrial $, and $ Anomaly $.
Further, it is described what user statistics are recorded in the benchmark study and what the process of participating in the benchmark study was like.

%%% Chapter 5
% DL-based methods vs. classical methods
The results of the benchmark study and further experiments are evaluated in Chapter \ref{ord:ch5}.
To begin the research question posed in Chapter \ref{ord:ch1} is examined, if interactive segmentation methods based on \gls{dl} improve the labeling process compared to methods without \gls{dl}.
The comparison is based on the benchmark results and takes the accuracy and the annotation time into account as presented in Section \ref{ord:ch5:sec1}.
The accuracy of the \gls{dextr} method performed best, while the the \gls{iog} method performs worst, as presented.
Therefore, in this regard neither classical or \gls{dl} based methods are advantageous, but as single method \gls{dextr} is superior.
With respect to annotation time, it could be clearly established that \gls{dl} based methods are significantly faster in application than classical methods.

% Generalization over domains
Further, the generalization capabilities over several image domains are evaluated in Section \ref{ord:ch5:sec2_generalization_image_domains}.
It is found that the $ IoU $ is about the same for over the image domains, except for the domain $ Anomaly $, what is caused by the basic nature of anomaly images.
In terms of the annotation time, it is discovered that the domains $ Standard $ and $ Urban $ are more elaborate to annotate, which is evident across all methods.

% Generalization over users
In Section \ref{ord:ch5:sec_3_generalization_user}, the generalization over benchmark participants is evaluated.
It is shown, that the annotation time for the \gls{dextr} and \gls{iog} method contains less variance over the users.
In relation to accuracy the \gls{dextr} method performs the most constant over various users.
It experienced, that the annotation time highly depends on the individual user.
In general, the \gls{dextr} method generalizes best over various image domains and user based on the results from the user benchmark.

Further on, in Section \ref{ord:ch5:sec4_survey} the survey filled out by the benchmark participants and their results is presented.
The outcome of this small survey indicates, that a pleasant user experience requires a good performing method and vice versa.

Last, in Section \ref{ord:ch5:sec5_retrain} the statement from Manisis \etal is examined, that \gls{dl} models trained on \gls{dextr} annotations perform equally as models trained with the original \gls{gt} \cite{Man18-DEXTR}.
The experiments of this thesis show, that this statement can be confirmed in a limited way.
The different models perform similar based on the \gls{map}, but the models trained on the \gls{dextr} annotations perform significantly worse for high \gls{iou} values. 


% Void Pixel 
An additional insight is about the use of \gls{vp} in the PASCAL \gls{voc} dataset \cite{Eve20-PascalVOC}.
As presented in Table \ref{tab:ch5:tests_on_datasets} the \gls{miou} differs significantly with and without the use of \gls{vp}.
By presenting the results with \gls{vp}, unrealistically high expectations are created that cannot be met on real world data or datasets without \gls{vp}.
The inclusion of the results with \gls{vp} would lead to more transparently and awareness of the purpose of \gls{vp}.

% Final tought
% Antwort auf Research Frage -> if interactive segmentation methods based on \gls{dl} improve the process of labeling based on the annotation time and accuracy.
In conclusion, it can be said that interactive segmentation methods based on \gls{dl} outperform  methods without \gls{dl} based on the annotation time, while in terms of accuracy, both methods are competitive on a similar level.
The generalization capabilities over different image domains are similar, while \gls{dl} based methods generalize better over various users, which is most probably caused by the guided design of the user interaction.
Thereby, the \gls{dextr} method with the direct and simple user interactions performs best on the benchmark study.


%%% Future Work
 