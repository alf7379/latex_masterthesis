% !TeX root = ../../main.tex
% Add the above to each chapter to make compiling the PDF easier in some editors.

\chapter{Benchmark Evaluation}\label{ord:ch5}

All user interactions were recorded during their participation in the benchmark. 
In this chapter, we evaluate this acquired data.
To judge if the differences of methods are significant statistical methods are applied, which are in general supported by the large amount of data with \getNumberBenchmarkAnnotations annotations from \getNumberBenchmarkRuns runs on the benchmark study.
To illustrate how annotations from the benchmark study look like Figure \ref{fig:ch5:segmentation_examples} provides few exemplary annotations, while a detailed insight of 27 predictions from the methods watershed, \gls{dextr} and \gls{iog} is given in Figure \ref{fig:appendix_model_predictions}.

%TODO Bitte die gesamte Arbeit nochmal durchlesen und auf Plural vs. Singular bei den Verben und den Substantiven achten! Solche Fehler reduzieren die Lesbarkeit und schmälern die schönen Ergebnisse!
%TODO beispielhafte grafic that shows annotations
\begin{figure} [b]
	\centering
	\begin{subfigure}[t]{0.3\textwidth}
		\centering
		\includegraphics[width=\textwidth]{figures/placeholder.png}
		\caption{
			tbd
		}
	\end{subfigure}
	\hfill
	\begin{subfigure}[t]{0.3\textwidth}
		\centering
		\includegraphics[width=\textwidth]{figures/placeholder.png}
		\caption{
			tbd
		}
	\end{subfigure}
	\hfill
	\begin{subfigure}[t]{0.3\textwidth}
		\centering
		\includegraphics[width=\textwidth]{figures/placeholder.png}
		\caption{
			tbd
		}
	\end{subfigure}
	\caption[tbd]{
		tbd
	} \label{fig:ch5:segmentation_examples}
\end{figure}


% !TeX root = ../../main.tex
% Add the above to each chapter to make compiling the PDF easier in some editors.

\section{Benchmark Description}\label{ord:ch5:sec1}

\subsection{Polygon Drawing}\label{ord:ch5:sec1:subsec1}
Lorem ipsum dolor sit amet, consetetur sadipscing elitr, sed diam nonumy eirmod tempor invidunt ut labore et dolore magna aliquyam erat, sed diam voluptua. At vero eos et accusam et justo duo dolores et ea rebum. Stet clita kasd gubergren, no sea takimata sanctus est Lorem ipsum dolor sit amet. Lorem ipsum dolor sit amet, consetetur sadipscing elitr, sed diam nonumy eirmod tempor invidunt ut labore et dolore magna aliquyam erat, sed diam voluptua. At vero eos et accusam et justo duo dolores et ea rebum. Stet clita kasd gubergren, no sea takimata sanctus est Lorem ipsum dolor sit amet.

\subsection{Structure}\label{ord:ch5:sec1:subsec2}
Lorem ipsum dolor sit amet, consetetur sadipscing elitr, sed diam nonumy eirmod tempor invidunt ut labore et dolore magna aliquyam erat, sed diam voluptua. At vero eos et accusam et justo duo dolores et ea rebum. Stet clita kasd gubergren, no sea takimata sanctus est Lorem ipsum dolor sit amet. Lorem ipsum dolor sit amet, consetetur sadipscing elitr, sed diam nonumy eirmod tempor invidunt ut labore et dolore magna aliquyam erat, sed diam voluptua. At vero eos et accusam et justo duo dolores et ea rebum. Stet clita kasd gubergren, no sea takimata sanctus est Lorem ipsum dolor sit amet.

% !TeX root = ../../main.tex
% Add the above to each chapter to make compiling the PDF easier in some editors.

\section{Generalization Over Domains} \label{ord:ch5:sec2_generalization_image_domains}
% RE-1467
% Motivation - get insights of the performance on other domains (anomaly and industrial)
The ability of methods to generalize well may be crucial to their success in application.
For \gls{dl} methods in general the generalization capability is mostly based on the dataset used for training.
The \gls{dextr} and \gls{iog} method are trained on a combination of PASCAL \gls{voc} and COCO.
These datasets mostly cover general objects and do not contain images from special domains.
In order to examine the method's generalization capabilities, they are evaluated over different domains and datasets.
First, the generalization capabilities of the four benchmark methods are evaluated over image domains from the benchmark study.
Thereby, the performance of $ IoU $ and $ time $ are examined.
Second, simulations are used to evaluate the $ IoU $ of the \gls{dextr} and \gls{iog} method on various datasets.

\subsection{Generalization Over Domains}
% Each method is evaluated over the four domains
In the following it is evaluated how well the four benchmark methods generalize over the four image domains: $ standard $, $ urban $, $ industrial $, and $ anomaly $.
Thereby, the evaluation of the generalization capabilities is twofold based on the $ IoU $ and $ time $.

% IoU
\subsubsection{IoU}
The generalization over the accuracy measured by the $ IoU $ is examined.
The accuracy per method and domain is visualized in Figure \ref{fig:ch5:sec2:methods_over_domain_iou}.

\begin{figure}[h!]
	\centering
	\includegraphics[width=\textwidth]{figures/chap52_iou_methods_over_domains_boxplot.png}
	\caption [Box plots of image domains and methods on  $ IoU $]{
		Box plots of $ IoU $ based on the image domains and benchmark methods.
		The $ IoU $ for the domain $ anomaly $ differs strongly for all methods, which can be explained by the different nature of this domain.
		It seems, that for the other domains all methods perform on a approximately similar level.
	}\label{fig:ch5:sec2:methods_over_domain_iou}
\end{figure}

% Performance of Anomaly
It is immediately noticeable that the \gls{iou} in the domain $ anomaly $ is significantly worse for each method.
This is reasonable due to the different nature of the annotations from this domain.
It was experienced in the review of the user annotations, that for the domain $ anomaly $ the users often have a varying understanding what belongs to the anomaly and what should be labeled.
These ambiguities create room for misinterpretations of the anomaly, which in turn affects the $ IoU $.
However, it may be noted that the \gls{dextr} method performs best on the domain $ anomaly $ most probably due to the strong and direct type of guidance provided by the extreme points.

% Performance of , urban, and 
Further, the four methods deliver comparatively similar results on the three other image domains $ standard $, $ urban $, and $ industrial $.
Thereby, the Kruskal-Wallis test is applied with the $ H_{0} $ already excluding the domain $ anomaly $, because it already can be seen in Figure \ref{fig:ch5:sec2:methods_over_domain_iou}, that $ IoU_{anomaly} $ strongly deviates.
The details of the test are presented in Table \ref{tab:ch5:tests_on_methods}.
%Only for the polygon method $ H_{0} $ is rejected at $ \alpha = 1\% $, while for the other methods $ H_{0} $ is narrowly accepted.
However, it is important to highlight, that this results strongly depends on the chosen significance level $ \alpha $.
The close decisions diminishes the significance of the statistical analysis and state that there is no clear decision as here.

However, it may be said that for the methods watershed, \gls{dextr}, and \gls{iog} the accuracy does not differs strongly over the three image domains $ standard $, $ urban $, and $ industrial $.
From this it can be concluded that these methods sufficiently generalize over different domains.
It is notable, that the \gls{dextr} and \gls{iog} method do not necessarily perform better on the domain $ standard $, even though they have been explicitly trained with this type of images. 
This also confirms the results from \cite{Man18-DEXTR} and \cite{Zha20-IOG}, which emphasize the performance of their methods across domains.
Although in general, an improvement of performance could be still possible if the model would be trained on the same type of images that are used for the evaluation.
This is strongly assumed, but not pursued further within the scope of this work.

\begin{table}[h!]
	\centering
	\begin{tabular}{l|p{25mm} p{25mm} p{25mm} p{25mm}}
		\toprule 		
		& \centering $ Polygon $	& \centering $ Watershed $ 	& \centering $ DEXTR $ 	& \multicolumn{1}{c}{$ IOG $}	\\
		\midrule
		$ n_{annots} $	& \centering 81				& \centering 98				& \centering 87			& \multicolumn{1}{c}{81}  		\\
		$ H_{0} $		& \multicolumn{4}{c}{$ med \left( IoU_{standard} \right) = med \left( IoU_{urban} \right) = med \left( IoU_{industrial} \right)$}  \\  
		$ H_{A} $		& \multicolumn{4}{c}{$ H_{0} = false $}  \\ 	
		$ \alpha $		& \centering $ 1\% $ 		& \centering $ 1\% $ 		& \centering $ 1\% $ 	& \multicolumn{1}{c}{$ 1\% $} 	\\ 	
		Statistic		& \centering 12.5154		& \centering 6.3355      	& \centering 9.0211		& \multicolumn{1}{c}{1.3372}  	\\ 
		$ \textnormal{\textit{p-value}} $
		& \centering 0.0019			& \centering 0.0421 		& \centering 0.0109		& \multicolumn{1}{c}{0.5124}	\\
		$ H_{0} $		& \centering rejected 		& \centering accepted	  	& \centering accepted 	& \multicolumn{1}{c}{accepted}  \\ 										
		\bottomrule
	\end{tabular}
	\caption[Kruskal-Wallis test performed on methods over domains]{
		For each benchmark method the Kruskal-Wallis test is applied to compare the $ IoU $ over the benchmark domains $ standard $, $ urban $, and $ industrial $.
		The domain $ anomaly $ is not included in this test, because it already can be seen in Figure \ref{fig:ch5:sec2:methods_over_domain_iou}, that the $ IoU $ differs for this domain.
		All tests are performed on the same $ H_{0} $ and $ H_{A} $.
		The number of annotations $ n_{annots} $ varies, because for each method the Kruskal-Wallis test requires the same size for the factors.
	}\label{tab:ch5:tests_on_methods}
\end{table}

% Time
\subsubsection{Time}
Further, the method's generalization capabilities are evaluated based on $ time $.
An overview of the annotation time per method and domain is provided by the box plots in Figure \ref{fig:ch5:sec2:methods_over_domain_time}.
% For Polygon and Watershed the domains  and urban require much time while the  and Anomaly are fast.
It is easy to see that more time is spent on the domains $ standard $ and $ urban $ than on the domains $ industrial $ and $ anomaly $.
This probably follows from the fact that the objects from domains $ standard $ and $ urban $ are more complicated and therefore require more annotation time.
% This behaviour may also be observed for DEXTR and IOG but not this extreme -> good generalization.
This behavior is extreme for watershed method and decrease continuously across the methods polygon, \Gls{iog}, and \gls{dextr}.

% Noteable in general is the larger annotation time for polygon and watershed.
In general, this demonstrates that the longer annotation time of polygon and watershed spans across the image domains.
It can be said that the \gls{dextr} and \gls{iog} method also generalize well based on the annotation time, since the values for $ time $ are in a similar range over the image domains.

\begin{figure}[h!]
	\centering
	\includegraphics[width=\textwidth]{figures/chap52_time_methods_over_domains_boxplot.png}
	\caption[Box plots of image domains and methods on  $ time $]{
		Box plots of $ time $ based on the image domains and benchmark methods.
		For the domains $ standard $ and $ urban $ the $ time $ is clearly higher, than for the other domains.
		This is extreme for the methods polygon and watershed.
		A difference may be noted for \gls{dextr} and \gls{iog} as well, but in general the generalization capabilities are sufficient.	
	} \label{fig:ch5:sec2:methods_over_domain_time}
\end{figure}

\begin{comment}
% Grouped by dataset with methods as factors
\subsection{Generalization Of Domains Over Methods} \label{ord:ch5:sec2:subsec1}
% For all methods - Kruskal-Wallis test (or even ANOVA) for the domains as factors
The sample is split up by the benchmark domains, introduced in Section \ref{ord:ch4:sec4}, and evaluated for each domain.
For this the Kruskal-Wallis test is applied to each split with the four benchmark methods as factors.
Due to the split by the four domains and the requirement of the same factor size by the Kruskal-Wallis test, the number of samples per factor is reduced to $n_{annots}$ in this test setup.
The details and results of the tests is shown in Table \ref{tab:ch5:tests_on_domains} and the corresponding box plot of illustrated in Figure \ref{fig:ch5:sec2:domains_box_plot}.

For the domains $ standard $ and $ urban $, which are similar to PASCAL \gls{voc} and COCO, $ H_{0} $ is accepted and the benchmark methods do not differ significantly.
In contrast, for the domains $ industrial $ and $ anomaly $ $ H_{0} $ is rejected and $ H_{A} $ accepted. Therefore, there is some significant difference between the benchmark methods.
\begin{table}[h!]
	\centering
	\begin{tabular}{l|p{25mm} p{25mm} p{25mm} p{25mm}}
		\toprule 		
								& \centering $ standard $	& \centering $ urban $  		& \centering $ industrial $ & \multicolumn{1}{c}{$ anomaly $} \\
		\midrule
		$ n_{annots} $			& \centering 105				& \centering 81				& \centering 351 			& \multicolumn{1}{c}{82}  \\
		$ H_{0} $				& \multicolumn{4}{c}{$ med \left( IoU_{polygon} \right) = med \left( IoU_{watershed} \right) = med \left( IoU_{IOG} \right) = med \left( IoU_{DEXTR} \right) $}  \\  
		$ H_{A} $		 		& \multicolumn{4}{c}{$ H_{0} = false $}  \\ 	
		$ \alpha $		 		& \centering $ 1\% $ 		& \centering $ 1\% $ 		  	& \centering $ 1\% $ 		& \multicolumn{1}{c}{$ 1\% $} 	\\ 	
		Statistic		 		& \centering 4.8254			& \centering 12.692	      		& \centering 30.3888			& \multicolumn{1}{c}{14.4029}  	\\ 
		$ \textnormal{\textit{p-value}} $
								& \centering 0.1850			& 0.0054 	& \centering $ 1.1 \cdot 10^{-6}$		& \multicolumn{1}{c}{0.0015}	\\
		$ H_{0} $		  		& \centering accepted 		& \centering rejected	  		& \centering rejected 		& \multicolumn{1}{c}{rejected}  \\ 										
		\bottomrule
	\end{tabular}
	\caption[Kruskal-Wallis tests over domains]{
		Four Kruskal-Wallis tests are applied for the domain scopes $ standard $, $ urban $, $ industrial $, and $ anomoaly $.
		The four benchmark methods are used as factors for each test.
		All tests are performed on the same $ H_{0} $, $ H_{A} $ and the same sample $X_{raw}$.
	}\label{tab:ch5:tests_on_domains}
\end{table}

The post analysis is performed by the DSCF test \cite{CF91-dscf}, which determines the factor, that differs significantly in the domains $ industrial $ and $ anomaly $.
As a result, for the domain $ standard $ only the \gls{iog} method gave evidence for a statistical significant decrease in performance.
In the domain $ anomaly $ two groups are formed, \gls{dextr} and Polygon achieve a significantly higher $ IoU $ than \gls{dextr} and Watershed.
Inside these two groups no significant difference occurs.
% Box plot with 16 "cols" - 4 methods x 4 domains
\begin{figure}
	\centering
	\includegraphics[width=0.9\textwidth]{figures/chap52_boxplot.png}
	\caption[Box plot IoU per domain and method]{
		Box plot of the $ IoU $ per image domain and benchmark method.
		In the domain $ anomaly $ all methods perform worse compared to the other domain, due to the special characteristics of this domain.
		For the domain $ anomaly $, $ med \left( IoU_{IOG} \right) $ is significantly lower, than for the other methods.
		The domains $ industrial $ and $ urban $ do not evidence a statistically significant difference between the methods.
	} \label{fig:ch5:sec2:domains_box_plot}
\end{figure}

Further, it was detected, that the domains $ urban $, $ industrial $, and $ anomaly $ do not differ significantly in their median value of the $ IoU $.
Only the domain scope $ anomaly $ differs greatly, which is reasonable due to the clearly different types of images.
It was experienced in the review of the user annotations, that the users often have a varying understanding what belongs to the anomaly, which should be labeled.
This ranges from different to misinterpretations, which also in an explanation for the general lower $ IoU $ in this domain.
The good performance of the polygon and \gls{dextr} method in the domain $ anomaly $ most probably is caused by the strong and direct type of guidance provided by the user interaction.

In conclusion it may be stated, that the examined benchmark methods mostly do not significantly vary over common domains, excluding the domain $ anomaly $.
Only for the \gls{iog} method did not generalize as well as the other methods on the domain $ standard $.
some comment
\end{comment}

\subsection{Generalization Over Datasets} \label{ord:ch5:sec2:subsec2}

In order to further evaluate the \gls{dextr} and \gls{iog} method, simulations are applied on various datasets from various domains.
To continue the evaluation of the generalization capabilities datasets from different domains are used.
First, the performance of the initial prediction over multiple datasets is presented, while later the refinement results are examined. 

\subsubsection{Initial Prediction}

% Define simulation settings
The different simulation setups from the original \gls{dextr} \cite{Man18-DEXTR} and \gls{iog} \cite{Zha20-IOG} have been implemented in HDevelop and unified to allow a fair comparison. 
Thereby, the \gls{gt} is used to simulate the user clicks on fore- and background to create the required model input.
It has to be highlighted, that this simulation setup only contains little deviation or randomness, similar to the original simulation setup from \gls{dextr} and \gls{iog}.
As a result, the simulated user clicks are almost perfect.
More realistic user clicks can be simulated by the inclusion of random deviation, which is presented in Subsection \ref{ord:ch5:sec3:subsec3}.

The performance of the \gls{dextr} and \gls{iog} method on various datasets is shown in Table \ref{tab:ch5:tests_on_datasets}.
It can be seen, that in general the \gls{iog} method performs best, except for the benchmark dataset, where \gls{dextr} achieves a higher \gls{miou}.
The superiority in the simulation of \gls{dextr} on this data set is consistent with the results from the real users from the benchmark study.

\begin{table}[h!]
	\centering
	\begin{tabular}{l|c c}
		\toprule 		
										& \multicolumn{2}{c}{mIoU}\\
										&  DEXTR 	& IOG		\\
		\midrule
		PASCAL (VP \cmark)				& 0.9103 	& 0.9251	\\
		PASCAL (VP \xmark)				& 0.7805	& 0.8086	\\
		Benchmark Dataset				& 0.8394 	& 0.8201	\\
		MVTec D2S						& 0.9297	& 0.9342	\\
		Segmentation and Counting		& 0.8280	& 0.8797 	\\							
		\bottomrule
	\end{tabular}
	\caption[Generalization of IOG and DEXTR]{
		The \gls{dextr} and \gls{iog} method are applied on multiple datasets to evaluate their performance over several domains.
		For the PASCAL \gls{voc} dataset the performance is measured with (\gls{vp} \cmark) and without (\gls{vp} \xmark) the application of the \gls{vp}.
	}\label{tab:ch5:tests_on_datasets}
\end{table}

The difference in accuracy for the PASCAL \gls{voc} dataset \cite{Eve20-PascalVOC} with and without the application of \gls{vp} is especially interesting.
The meaningfulness of \gls{vp} is already addressed in Subsection \ref{ord:ch2:sec2} and their effect can be observed in Table \ref{tab:ch5:tests_on_datasets}.
For \gls{dextr} the difference in \gls{miou} is around $ 0.13 $ and for \gls{iog} almost $ 0.12 $.
This represents a significant decrease in accuracy, which raises doubts how expressive the results with the applications of \gls{vp} are.
Especially problematic is that the \gls{vp} are located at the edge of an object, which is generally difficult to predict.
As a consequence, it may not be recognized if a model's weakness is the prediction of the edges.
Therefore, the use of \gls{vp} may lead to unrealistic expectations on the method's accuracy, which are mostly not achievable in real world application.
The presentation of the performance without \gls{vp} would lead to more transparency and awareness of this characteristic.

\subsubsection{Refinement Prediction}

% Show a table displaying the performance for various refinement clicks on multiple dataset
Last, the performance of the method with the application of refinement is evaluated on different datasets.
Thereby, for each method five refinement clicks were simulated.
For the \gls{iog} method, the refinement clicks can be on the fore- and background, while the \gls{dextr} method only allows refinement clicks on the border.
The refinement clicks are simulated using the \gls{gt}.
A click is set on the region where the previous prediction fails.

\begin{table}[h!]
	\centering
	\resizebox{\textwidth}{!}{
	\begin{tabular}{l l|c c c c c c}
		\toprule
				&						& \multicolumn{6}{c}{mIoU} \\
				& {$ n_{\textnormal{\textit{refine clicks}}} $} 
										& 0			& 1			& 2			& 3 		& 4 		& 5			\\
		\midrule
		DEXTR 	& PASCAL (VP \cmark)	& 0.9103	& 0.9119	& 0.9054	& 0.8986 	& 0.8916	& 0.8863	\\
			 	& PASCAL (VP \xmark)	& 0.7805	& 0.7894 	& 0.7893	& 0.7871 	& 0.7833	& 0.7807	\\
				& Benchmark				& 0.8394	& 0.8522 	& 0.8513	& 0.8504 	& 0.8512	& 0.8492	\\
				& MVTec D2S				& 0.9297	& 0.9416 	& 0.9436	& 0.9437 	& 0.9419	& 0.9397	\\
		\midrule
		IOG 	& PASCAL (VP \cmark)	& 0.9251	& 0.9359	& 0.9402	& 0.9436 	& 0.9449	& 0.9459	\\
				& PASCAL (VP \xmark)	& 0.8086	& 0.8184	& 0.8259	& 0.8313 	& 0.8351	& 0.8381	\\
				& Benchmark				& 0.8219	& 0.8396 	& 0.8508	& 0.8601	& 0.8669	& 0.8715	\\
				& MVTec D2S				& 0.9343	& 0.9358 	& 0.9424	& 0.9468 	& 0.9499	& 0.9523	\\
		\bottomrule
	\end{tabular}}
	\caption[Generalization of IOG and DEXTR refinement]{
		Performance of the \gls{dextr} and \gls{iog} method for different number of refinement clicks.
		At $ n_{\textnormal{\textit{refine clicks}}} = 0 $, no refinement click is set, which is equivalent to the initial prediction.
	}\label{tab:ch5:tests_on_refinement_datasets}
\end{table}

In Table \ref{tab:ch5:tests_on_refinement_datasets} the performance of the methods is presented for various refinement clicks.
For \gls{dextr} mostly only the first refinement clicks leads to a improvement, while the further clicks keep or decrease the \gls{miou}.
In contrast, the \gls{iog} method is able to profit from each refinement click and increase the \gls{miou}. 
From the ninth click, the improvement stagnates or worsens slightly as shown in Table \ref{tab:appendix_refinementclicks}.
It can be stated, that refinement process generalizes well and performs expected over several domains.

% !TeX root = ../../main.tex
% Add the above to each chapter to make compiling the PDF easier in some editors.

\section{Generalization over users}\label{ord:ch5:sec_3_generalization_user}

% RE-1468
This benchmark study strongly benefits from having real participants applying various methods.
Valuable data is collected, which is, however, challenging to interpret.
This is because participants often have different characteristics as levels of experience, understanding of accuracy, professional background, and motivation.
As a result, the interactive method may be handled differently and the obtained data may vary with respect to various users with different characteristics. 

% Ziel = Methode zu finden mit der alles User gut klarkommen und die über verschiedene user gute Resultate in Zeit und IoU erziehlt.
However, in order to find wide application, an interactive method must work consistently with diverse users.
This section examines the generalization capabilities of the benchmark methods across different users and, therefore, is the counterpart to the previous section.
% Motivation: generalization over user -> an interactive method really performs well and reliable if it delivers good results for all possible users.




\subsection{Benchmark Participants Evaluation}\label{ord:ch5:sec3:subsec1}

In the following the performance of the methods is evaluated over the \getNumberBenchmarkParticipants participants.
In contrast to the previous evaluation, here the benchmark runs performed by the colleagues, who are working on this topic, are excluded.
This is done intentionally, in order to not distort the evaluation, since the users are explicitly evaluated here and the data obtained from most experienced users would be biased.

In order to evaluate the generalization capabilities across multiple users, first statistical key figures as the mean and standard deviation $ \sigma $ are evaluated.
The combined mean and $ \sigma $ for the participants are presented in Table \ref{tab:ch5:all_benchmark_users_varaince}.
% For IoU low std-dev, but for time vergleichsweise hoch 
It can be observed, that $ \sigma_{IoU} $ is very similar for the benchmark methods and only ranges from $ \sigma_{IoU} = 0.1187 $ for \gls{dextr} to $ \sigma_{IoU} = 0.1669 $ for watershed.
In contrast $ time $ shows a strong deviation, with $ \sigma_{time} = 15.0977 $ for \gls{dextr} the deviation is less than half as large as for polygon with $ \sigma_{time} = 39.0214 $.
The mean value is presented to put $ \sigma $ into context.
A graphical illustration of the deviation of $ IoU $ and $ time $ is already presented in the box plots from Figure \ref{fig:ch5:sec1:iou_box_plot} and \ref{fig:ch5:sec1:time_box_plot}.
Further, these results support the conclusion of Subsection \ref{ord:ch5:sec1:subsec2} and \ref{ord:ch5:sec1:subsec3}, which state that in terms of $ IoU $ the methods perform approximately equal, while $ \overline{time} $ and $ med(time) $ do differ significantly between the \gls{dl} based and classical benchmark methods.
% Calculate the variance for time and iou.
\begin{table}[h!]
	\centering
	\begin{tabular}{l|c c c c}
		\toprule 		
			 				& $ Polygon $  	& $ Watershed $ 	& $ DEXTR $ 	& $ IOG $	\\
		\midrule
		\rule{0pt}{3ex}%  EXTRA vertical height  
		$ \overline{IoU} $	& 0.8066 		& 0.8166		 	& 0.8543		& 0.8020	\\
		\rule{0pt}{3ex}%  EXTRA vertical height  
		$ \sigma_{IoU} $	& 0.1287 		& 0.1669		 	& 0.1187 		& 0.1559	\\
		\rule{0pt}{3ex}%  EXTRA vertical height  
		$ \overline{time} $	& 34.399 		& 45.076			& 14.542 		& 18.059	\\
		\rule{0pt}{3ex}%  EXTRA vertical height  
		$ \sigma_{time} $	& 39.021 		& 37.796			& 15.098 		& 17.383	\\
		\bottomrule
	\end{tabular}
	\caption[Mean and standard deviation of the benchmark methods]{
		Presentation of the mean and standard deviation $ \sigma $ for $ IoU $ and $ time $.
		It can be seen that $ \sigma_{IoU} $ stays almost constant over the benchmark methods.
		In contrast, $ \sigma_{time} $ differs between the methods.
		The mean value was given for IoU and time to better interpret $ \sigma $.
	}\label{tab:ch5:all_benchmark_users_varaince}	
\end{table}
% TODO Mir fehlt hier irgendwie eine genauere Angabe was das für Werte sind. In diesem Kapitel geht es doch um die User?     Irgendwo (vllt auch als Formel) sollte gezeigt werden, dass hier ein Mittel über die User-weisen Mittel (mean_mean_IoU_per_user_per_method) gebildet wird.     Genauso ist das wohl die mittlere Varianz (gemittelt über die User), oder?


To get even deeper insights about the generalization across different users, in Figure \ref{fig:ch5:sec3:all_benchmark} box plots for $ IoU $ and $ time $ are shown for all participants individually.
Here, no statistical tests for equality within the methods were performed, because the visualization already indicates that there are strong deviations between the user for each method.
The variations between the individual users are present in all methods and most likely due to the varying characteristics of the users as introduced in the beginning of this section.
% \eg different experience levels, how , and how well a user got along with the method.

However, it can be stated that for $ IoU $ \gls{dextr} has the smallest deviation and is the most constant method over all participants, while the other methods still perform similar.

% Box plot of all BenchmarkParticipants (not experienced user aka me)
\begin{figure} 
	\centering
	\begin{subfigure}[t]{1.0\textwidth}
		\centering
		\includegraphics[width=\textwidth]{figures/chap53_all_users_iou.png}
		\caption{
			The \gls{dextr} delivers mostly constant $ IoU $ values with some outliers, while the other methods are mostly characterized by irregularities.
		}\label{fig:ch5:sec3:all_benchmark_iou}
	\end{subfigure}
	\\
	\begin{subfigure}[t]{1.0\textwidth}
		\centering
		\includegraphics[width=\textwidth]{figures/chap53_all_users_time.png}
		\caption{
			For $ time $ the \gls{dextr} method convinces with a mostly constant performance, while the \gls{iog} method has two users who are the exception to the consistency.
			In contrast for the polygon and watershed method consistency over various user is not given.			
		} \label{fig:ch5:sec3:all_benchmark_time}
	\end{subfigure}
	\caption[Box plots of benchmark participants on $ IoU $ and $ time$.]{		
		The box plots of $ IoU $ and $ time $ over the \getNumberBenchmarkParticipants benchmark participants.
		The single box plots indicate how the methods performed over the single users, the order of the users is the same for both figures. 
		Large differences between the box plots within a method indicate poor generalization capabilities and vice versa.
	}\label{fig:ch5:sec3:all_benchmark}
\end{figure}

In terms of $ time $ the \gls{dl} based methods show less variance over the users than the classical methods.
The \gls{dextr} method performs best, while the \gls{iog} method has two bigger outliers.
The difference between \gls{dl} based and classical methods is partly caused by the design of the methods, which guides the user more or less.
On the one side, the polygon and watershed method do little to guide the user,
but give the user a lot of freedom when performing the label task.
This room for interpretation is used differently by the users, which leads to different label times and, therefore, higher variance.
On the other side, the \gls{dextr} and \gls{iog} method provide strong guidance by defining the amount of clicks required for an initial prediction.
Although the application of refinement is still possible, this prevents users from spending too much time on an initial prediction.
This strong guidance of the user while applying the method leads to a more constant annotation time and thereby a lower variance in $ time $.
% polygon and watershed are more freely

In conclusion, for $IoU$ the \gls{dextr} method generalizes best over multiple users, while the other models do no achieve a constant performance over various users.
For $ time $ it is observed, that methods with strong guidance as \gls{dextr} and \gls{iog} are more constant in $ time $, than the polygon and watershed method, that are designed more open and, therefore, allow higher variance.


\subsection{Two Experienced Users} \label{ord:ch5:sec3:subsec2_cmo_afe}

Normally, the participants have labeled only a part of the benchmark images. 
In the following, eight benchmark runs are evaluated where all benchmark images were labeled by two users who have more experience working on this topic.
Although only two different users are compared with each other, the significance is given, since all images of the benchmark were labeled.
However, it must be mentioned that the experience level of both users is not the same, because $ \textnormal{\textit{User 2}} $ is more trained than $ \textnormal{\textit{User 1}} $.

% No significant difference in the IoU between the two user could be detected -> generalize well over various users (few user, large data)
For the $ IoU $, the box plots in Figure \ref{fig:ch5:sec3:cmo_afe_iou} indicate, that the $ IoU $ is more or less equal fors $ \textnormal{\textit{User 1}} $ and $ \textnormal{\textit{User 2}} $ for the four benchmark methods.
The equality of the obtained values from both users are statistically confirmed by the Kruskal-Wallis test, the details of the tests are presented in Table \ref{tab:appendix:afe_cmo_kruskal_wallis_iou}.

An equivalent analysis was performed for $ time $, as presented in Figure \ref{fig:ch5:sec3:cmo_afe_time}.
Interestingly, here the annotation time between $ \textnormal{\textit{User 1}} $ and $ \textnormal{\textit{User 2}} $ differs strongly for polygon and watershed, while the annotation time is very similar for \gls{dextr} and \gls{iog}.
The statistical relevance of this observation is again confirmed by the Kruskal-Wallis test, presented in detail in Table \ref{tab:appendix:afe_cmo_kruskal_wallis_time}.

\begin{figure} [h!]
 	\centering
 	\begin{subfigure}[t]{0.45\textwidth}
 		\centering
 		\includegraphics[width=\textwidth]{figures/chap53_afe_cmo_iou.png}
 		\caption{
 			Box plots of the $ IoU $ for the four benchmark methods of $ \textnormal{\textit{User 1}} $ and $ \textnormal{\textit{User 2}} $.
 			The visualization does not indicate any significant difference in $ IoU $.
 		} \label{fig:ch5:sec3:cmo_afe_iou}
 	\end{subfigure}
 	\hfill
 	\begin{subfigure}[t]{0.45\textwidth}
 		\centering
 		\includegraphics[width=\textwidth]{figures/chap53_afe_cmo_time.png}
 		\caption{
 			Box plots of the $ time $ for the four benchmark methods of $ \textnormal{\textit{User 1}} $ and $ \textnormal{\textit{User 2}} $.
 			A significant difference in $ time $ for the polygon and watershed is shown, while the \gls{dextr} and \gls{iog} methods perform similar.
 		}\label{fig:ch5:sec3:cmo_afe_time}
 	\end{subfigure}
 	\caption[Box plots of two experienced user on $ IoU $ and $ time$.]{		
 		These two figures show the comparison of the two users over the benchmark methods based on $ IoU $ and $ time $.
 		The hypothesis based on the graphical visualizations are statistically supported by the Kruskal-Wallis test as presented in detail in Table \ref{tab:appendix:afe_cmo_kruskal_wallis_iou} and \ref{tab:appendix:afe_cmo_kruskal_wallis_time}.
 	}\label{fig:ch5:sec3:cmo_afe}
\end{figure}

% User 2 more experienced that User 1
This large difference for Polygon and Watershed was not expected, but it emphasizes again how large the variance can be between different users who achieve the same performance.
This difference may be partly caused by the different level of experience of the two users.
Anyway, it is not surprising that this occurs for the classical methods, that are more freely interpretable and, therefore, leave the user a more open application.
Rather this supports the conclusion of the previous subsection, that methods with stronger guidance perform more consistent in the annotation time.


\subsection{Simulations With Different Click Patterns}\label{ord:ch5:sec3:subsec3}
% RE-1468

% Simulations are easily scalable, faster and cheaper than the acquisition of manual clicks from real users.
% Second, in a simulation no variance occurs between the set clicks of various users, if the.
% Third, simulations have the possibility to effortless create various click patterns, that \eg vary the set click by a random offset, in order to simulate a various types of user behavior.
%On the other hand, simulations are only capable to replicate the user's behavior to a certain extent.
% Further, the involvement of human users is especially important for methods, which performance depends on user interactions.


In order to gain deeper insights on the functionality of the \gls{dextr} and \gls{iog} method the simulation setup is modified.
% Motivation - Nachahmung von unterschiedlichen Usertypen durch unterschiedliche Genauigkeit bei der Klick-Simulation.
The altered simulation setup uses different levels of accuracy, to simulate more realistic user clicks.
% Simulation of with different permutation / deviation / - simulation of different user

The level of accuracy is defined by a deviation of maximal $ n_{deviation} $ \Unit{px}.
In the range $ \left[-n_{deviation}, \dots, 0, \dots, n_{deviation} \right] $ two values are randomly selected and added to the ideal row and column.
This random factor was added, to simulate the varying accuracy of single user clicks.
The smaller $ n_{deviation} $, the more accurate are the simulated clicks.
In the simulation first the ideal click position is calculated and further the random deviation is added.
This procedure is applied to the extreme points of the \gls{dextr} method and to the foreground and background clicks of the \gls{iog} method.
\begin{table}[h!]
	\centering	
	\resizebox{\textwidth}{!}{
	\begin{tabular}{l l|c c c c c c c}
		\toprule
				&						& \multicolumn{7}{c}{mIoU} \\
				& {$ n_{deviation} $} 	& 0	\Unit{px}	& 5	\Unit{px}	& 10 \Unit{px}	& 15 \Unit{px}	& 20 \Unit{px}	& 25 \Unit{px}	& 30 \Unit{px}	\\
		\midrule
		DEXTR 	& PASCAL (VP \cmark)	& 0.9103	& 0.8323	& 0.7626	& 0.7032 	& 0.6479	& 0.6047	& 0.5626		\\
				& PASCAL (VP \xmark)	& 0.7807	& 0.7523	& 0.7019	& 0.6543 	& 0.6085	& 0.5695	& 0.5317		\\
				& Benchmark				& 0.8414	& 0.7743	& 0.6887	& 0.6240 	& 0.5579	& 0.5027	& 0.4892		\\
		\midrule
		IOG 	& PASCAL (VP \cmark)	& 0.9267	& 0.8846	& 0.8092	& 0.7352 	& 0.6578	& 0.5953 	& 0.5457		\\
				& PASCAL (VP \xmark)	& 0.8081	& 0.7720	& 0.7099	& 0.6476 	& 0.5979	& 0.5337	& 0.4873		\\
				& Benchmark				& 0.8219	& 0.7865	& 0.7173	& 0.6027 	& 0.5268	& 0.4493 	& 0.4058		\\
		\bottomrule
	\end{tabular}}
	\caption[Simulations with different click patterns]{
		Simulations of the \gls{dextr} and \gls{iog} method with user clicks, that are simulated with varying degrees of accuracy.
		The parameter $ n_{deviation} $ states the maximal possible deviation from the optimal point, in order to mimic different types of users.
		As expected, the performance decreases with increasing deviation in the simulated user clicks.
	}\label{tab:ch5:simulation_various_click_patterns}
\end{table}

The performance constantly decreases with higher $ n_{deviation} $, as demonstrated in Table \ref{tab:ch5:simulation_various_click_patterns}. 
Even a comparable small deviation of maximal 5 \Unit{px} leads a notable drop in performance \eg from $ mIoU = 0.9103 $ to 0.8323 for \gls{iog} on the PASCAL \gls{voc} dataset with the application of \gls{vp}.
This factor needs to be taken into account, if these methods are applied by real users.





% !TeX root = ../../main.tex
% Add the above to each chapter to make compiling the PDF easier in some editors.

\section{Survey Evaluation}\label{ord:ch5:sec4_survey}
% RE-1469
After participating in the benchmark study 17 users also filled out the post survey.
This survey consists out of five questions, to gain deeper insights on the user experience.
The questions and their results are presented in \ref{fig:ch5:sec4:suvery}.
% Soft Factors
% Useability

\begin{figure} [h!]
	\centering
	\begin{subfigure}[t]{0.48\textwidth}
		\centering
		\includegraphics[width=\textwidth]{figures/chap54_q1.png}
		\caption{
			Question 1: \textit{The result / prediction masks of the method were as I expected them to be.}
		} \label{fig:ch5:sec4:q1}
	\end{subfigure}
	\hfill
	\begin{subfigure}[t]{0.48\textwidth}
		\centering
		\includegraphics[width=\textwidth]{figures/chap54_q2.png}
		\caption{
			Question 2: \textit{I was satisfied with the result / prediction masks.}
		} \label{fig:ch5:sec4:q2}
	\end{subfigure}
	\\
	\begin{subfigure}[t]{0.48\textwidth}
		\centering
		\includegraphics[width=\textwidth]{figures/chap54_q3.png}
		\caption{
			Question 3: \textit{With more experience / time in the labeltool I became better at handling the method and applying it.}
		} \label{fig:ch5:sec4:q3}
	\end{subfigure}
	\hfill
	\begin{subfigure}[t]{0.48\textwidth}
		\centering
		\includegraphics[width=\textwidth]{figures/chap54_q5.png}
		\caption{
			Question 4: \textit{It was ‘fun’ / ‘exciting’ to use the method.}
		} \label{fig:ch5:sec4:q4}
	\end{subfigure}
	\\
	\begin{subfigure}[t]{0.48\textwidth}
		\centering
		\includegraphics[width=\textwidth]{figures/chap54_q4.png}
		\caption{
			Question 5: \textit{By manually drawing a polygon I could achieve a better result in the same time.}
		} \label{fig:ch5:sec4:q5}
	\end{subfigure}
	\caption [Watershed User Interaction]{
		Responses on the questions from the survey.
		The answer options for all question are \textit{Strongly Agree}, \textit{Agree}, \textit{Neutral}, \textit{Disagree}, and \textit{Strongly Disagree}.
		Overall, the \gls{dextr} method receives good evaluations from the users.
	} \label{fig:ch5:sec4:suvery}
\end{figure}

In order to be accepted by users a method needs to be understandable and create deterministic results.
When the method performs as intended, a pleasant user experience is created, while frustration is caused otherwise.
This is investigated in the presented questions, in question 1 and 2 where the satisfaction and expectation of the users are examined.
While in question 4 the pleasantness of the user experience is captured.
These questions support the statement from the beginning, since the methods that received comparatively bad reviews on questions 1 and 2 are also not favored from the users in question 4.
Here the \gls{dl} based methods received good feedback.
This reasons in their little user interaction and excitement of applying novel methods.
Further, in question 3 and 5 it was researched whether the user was time efficient and improved over time.

In conclusion, based on this survey the \gls{dextr} method seems to provide the best user experience compared to the other benchmark methods.
Manual methods like polygon drawing or watershed also serve good results, but are not time efficient and pleasant to apply, compared to the \gls{dextr} and \gls{iog} method.
In this context, the answers suggest that good performance requires good user experience and vice versa.


% !TeX root = ../../main.tex
% Add the above to each chapter to make compiling the PDF easier in some editors.

\section{Suitability Of DEXTR and IOG Annotations For Training} \label{ord:ch5:sec5_retrain}
% RE-1470
In this section it is examined if annotations created by the \gls{dextr} and \gls{iog} method are suitable to train a new \gls{dl} model for semantic segmentation. %TODO or instance segmentation?
In the following the experimental setup is described in three stages: creation of annotations, model training, and model evaluation.
 
% Create annotations.
As first step, simulations are applied to create the required user input for the \gls{dextr} and \gls{iog} method.
The resulting predictions from the models are treated as new labels $ \textbf{Y}_{new} $. 
The simulation is performed with one refinement click for \gls{dextr} and four refinement clicks for \gls{iog}.
Further, the simulation setup is without mayor randomization, as described in Subsection \ref{ord:ch5:sec2:subsec2}.
The simulation is performed on the training split of the MVTec \gls{d2s} dataset \cite{Paddo18-D2S}, the test split is not altered.
Thereby, two variants of the \gls{d2s} dataset are used, in the training split the normal version of has 4380 images and 6900 annotations, while the augmented version contains 14380 images and 83337 annotations.
The goodness of the newly created annotations $ \textbf{Y}_{new} $ is at least at 0.91 \gls{miou}, as presented in Table \ref{tab:ch5:results_annot_usability}.

% Train model.
In the second step, the new annotations $ \textbf{Y}_{new} $ of the train split, are used to train semantic segmentation models $ m $.
As backbone of a model the Mask-RCNN \cite{He17-MaskR-CNN} is used, which is pretrained on the COCO dataset \cite{Lin14-Coco}.
All trainings are performed with the same settings, to enable a fair comparison.
A training lasts $ n_{epochs} = 30 $ epochs, with the learning rate $ \lambda = 0.001 $ and \gls{sgd} optimizer \cite{Ruder16-SGD}.
To establish a benchmark for comparison, also two models are trained on the original \gls{gt} of the normal and augmented \gls{d2s} dataset.
The models itself do not contain any novelty, but are just used to be trained with the new annotations $ \textbf{Y}_{new} $.

% Eval model.
Third, the performance of the created model is evaluated by the \gls{miou} and the \gls{ap} at different levels if \gls{iou}, as presented in Table \ref{tab:ch5:results_annot_usability}.
Based on the \gls{map}, the models trained on the \gls{dextr} and \gls{iog} annotations, perform approximately as good as the model trained on the original \gls{gt}.
This approximately applies to both \gls{d2s} dataset, but for the augmented \gls{d2s} dataset the model trained on the original \gls{gt} performs slightly better.

%For the \gls{d2s} dataset the model trained on the \gls{dextr} annotations even achieves a \gls{map} $ mAP = 0.4459 $ and therefore is slightly better than the model trained on the original \gls{gt} with $ mAP = 0.4398 $.
However, differences between the models can be observed when considering the IoU for certain degrees of IoU.
The \gls{ap} for annotations at $ IoU < 0.75 $ are at an equal level for all methods.
From there on, the \gls{ap} tends to decrease for a higher \gls{iou}.
This observation is extreme for $ IoU > 0.95 $. 
Exemplary for the models trained on the normal data set, the original \gls{gt} enables a performance of $ AP = 0.2198 $, while the performance of the \gls{dextr} and \gls{iog} model dropped to $ AP = 0.1549 $ respectively $ AP = 0.0384 $.
This effect also occurs for the augmented \gls{d2s} dataset.
From this follows that, the models trained on the \gls{dextr} and \gls{iog} annotations are not able to make prediction with a high level of detail.
The missing details in the model predictions, is probably caused by the \gls{dextr} and \gls{iog} annotations, which are also not perfect at the detail level.
So, in order to make create predictions with very high \gls{iou} and rich detail, the original \gls{gt} delivers better results.
%TODO reread the upper part once.


\begin{table}[h!]
	\centering
	\begin{tabular}{ ll|c c c|c c c}
		\toprule
								& Dataset				& \multicolumn{3}{c}{D2S} 			& \multicolumn{3}{c}{D2S augmented} \\		 
								& 						& GT		& DEXTR		& IOG		& GT		& DEXTR		& IOG		\\
		\midrule
		$ \textbf{Y}_{new} $ 	& mIoU 					& 1.0		& 0.9554	& 0.9271	& 1.0		& 0.9102	& 0.9209	\\
		\midrule
		$ m $					& mAP 					& 0.4398	& 0.4459	& 0.4138	& 0.7581	& 0.7374	& 0.6993	\\
		
								& AP @ $ IoU >= 0.5 $	& 0.4987	& 0.5221	& 0.5179	& 0.8574	& 0.8572	& 0.8543	\\
								& AP @ $ IoU = 0.55 $ 	& 0.4926	& 0.5145	& 0.5135	& 0.8565	& 0.8518	& 0.8502	\\
								& AP @ $ IoU = 0.6 $	& 0.4859	& 0.5063	& 0.5083	& 0.8526	& 0.8471	& 0.8451	\\
								& AP @ $ IoU = 0.65 $ 	& 0.4834	& 0.5035	& 0.5038	& 0.8476	& 0.8419	& 0.8343	\\
								& AP @ $ IoU = 0.7 $	& 0.4788	& 0.4967	& 0.4955	& 0.8425	& 0.8336	& 0.8162	\\
								& AP @ $ IoU = 0.75 $ 	& 0.4685	& 0.4876	& 0.4715	& 0.8276	& 0.8188	& 0.7899	\\
								& AP @ $ IoU = 0.8 $	& 0.4540	& 0.4687	& 0.4381	& 0.8016	& 0.7849	& 0.7370	\\
								& AP @ $ IoU = 0.85 $ 	& 0.4293	& 0.4401	& 0.3850	& 0.7641	& 0.7328	& 0.6718	\\
								& AP @ $ IoU = 0.9 $	& 0.3868	& 0.3651	& 0.2662	& 0.6617	& 0.6008	& 0.5050	\\
								& AP @ $ IoU = 0.95 $ 	& 0.2198	& 0.1549	& 0.0384	& 0.2694	& 0.2047	& 0.0982	\\
		\bottomrule
	\end{tabular}
	\caption[Performance of models trained with DEXTR and IOG annotations]{
		Performance of the \gls{dextr} and \gls{iog} annotations and the trained models.
		This setup was performed on the \gls{d2s} and the augmented \gls{d2s} dataset.
		The performance of the annotations created by the \gls{dextr} and \gls{iog} model is measured by the \gls{miou}.
		For each dataset three models were trained based on the original \gls{gt}, the \gls{dextr}, and \gls{iog} annotations.
		The performance of the model is evaluated with the \gls{map} and the \gls{ap} for different \gls{iou} levels.		
	}\label{tab:ch5:results_annot_usability}
\end{table}
%TODO determine if $ IoU >= 0.5 $ ot $ IoU = 0.5 $

In conclusion, the statement from Manisis \etal, that 
Finally, the statement of Manisis \etal is reviewed again, which claims that models trained on \gls{dextr} annotations perform equally well \cite{Man18-DEXTR}.
It can be said, that the \gls{dextr} and \gls{iog} annotations are suitable to train a new model, which performs equivalent on a lower $ IoU$ level and the \gls{map}.
However, for a high \gls{iou} the model trained on the \gls{dextr} annotations performs a bit worse, while the one from \gls{iog} annotations performs significantly worse.
Depending on the required level of precision of the model, the \gls{dextr} and \gls{iog} annotations are suitable for training.

It has to be noted, that these experiments are only performed on one dataset and the underlying model was already pretrained.
So, no general evidence is provided, but the basic capabilities have been demonstrated.
% finetuning
This approach may be especially useful, in the field of \textit{transfer learning}, to finetune a pretrained model with new annotations created by interactive segmentation methods.

