% !TeX root = ../../main.tex
% Add the above to each chapter to make compiling the PDF easier in some editors.

\section{Survey Evaluation}\label{ord:ch5:sec4_survey}
% RE-1469
After participating in the benchmark study 17 users also filled out the post survey.
This survey consists out of five questions, to gain deeper insights on the user experience.
The questions and their results are presented in \ref{fig:ch5:sec4:suvery}.
% Soft Factors
% Useability

\begin{figure} [h!]
	\centering
	\begin{subfigure}[t]{0.48\textwidth}
		\centering
		\includegraphics[width=\textwidth]{figures/chap54_q1.png}
		\caption{
			Question 1: \textit{The result / prediction masks of the method were as I expected them to be.}
		} \label{fig:ch5:sec4:q1}
	\end{subfigure}
	\hfill
	\begin{subfigure}[t]{0.48\textwidth}
		\centering
		\includegraphics[width=\textwidth]{figures/chap54_q2.png}
		\caption{
			Question 2: \textit{I was satisfied with the result / prediction masks.}
		} \label{fig:ch5:sec4:q2}
	\end{subfigure}
	\\
	\begin{subfigure}[t]{0.48\textwidth}
		\centering
		\includegraphics[width=\textwidth]{figures/chap54_q3.png}
		\caption{
			Question 3: \textit{With more experience / time in the labeltool I became better at handling the method and applying it.}
		} \label{fig:ch5:sec4:q3}
	\end{subfigure}
	\hfill
	\begin{subfigure}[t]{0.48\textwidth}
		\centering
		\includegraphics[width=\textwidth]{figures/chap54_q5.png}
		\caption{
			Question 4: \textit{It was ‘fun’ / ‘exciting’ to use the method.}
		} \label{fig:ch5:sec4:q4}
	\end{subfigure}
	\\
	\begin{subfigure}[t]{0.48\textwidth}
		\centering
		\includegraphics[width=\textwidth]{figures/chap54_q4.png}
		\caption{
			Question 5: \textit{By manually drawing a polygon I could achieve a better result in the same time.}
		} \label{fig:ch5:sec4:q5}
	\end{subfigure}
	\caption [Watershed User Interaction]{
		Responses on the questions from the survey.
		The answer options for all question are \textit{Strongly Agree}, \textit{Agree}, \textit{Neutral}, \textit{Disagree}, and \textit{Strongly Disagree}.
		Overall, the \gls{dextr} method receives good evaluations from the users.
	} \label{fig:ch5:sec4:suvery}
\end{figure}

In order to be accepted by users a method needs to be understandable and create deterministic results.
When the method performs as intended, a pleasant user experience is created, while frustration is caused otherwise.
This is investigated in the presented questions, in question 1 and 2 where the satisfaction and expectation of the users are examined.
While in question 4 the pleasantness of the user experience is captured.
These questions support the statement from the beginning, since the methods that received comparatively bad reviews on questions 1 and 2 are also not favored from the users in question 4.
Here the \gls{dl} based methods received good feedback.
This reasons in their little user interaction and excitement of applying novel methods.
Further, in question 3 and 5 it was researched whether the user was time efficient and improved over time.

In conclusion, based on this survey the \gls{dextr} method seems to provide the best user experience compared to the other benchmark methods.
Manual methods like polygon drawing or watershed also serve good results, but are not time efficient and pleasant to apply, compared to the \gls{dextr} and \gls{iog} method.
In this context, the answers suggest that good performance requires good user experience and vice versa.

