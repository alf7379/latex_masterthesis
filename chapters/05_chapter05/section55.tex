% !TeX root = ../../main.tex
% Add the above to each chapter to make compiling the PDF easier in some editors.

\section{Suitability Of DEXTR and IOG Annotations For Training} \label{ord:ch5:sec5_retrain}
% RE-1470
In this section it is examined if annotations created by the \gls{dextr} and \gls{iog} method are suitable to train a new \gls{dl} model for semantic segmentation. %TODO or instance segmentation?
In the following the experimental setup is described in three stages: creation of annotations, model training, and model evaluation.
 
% Create annotations.
As first step, simulations are applied to create the required user input for the \gls{dextr} and \gls{iog} method.
The resulting predictions from the models are treated as new labels $ \textbf{Y}_{new} $. 
The simulation is performed with one refinement click for \gls{dextr} and four refinement clicks for \gls{iog}.
Further, the simulation setup is without mayor randomization, as described in Subsection \ref{ord:ch5:sec2:subsec2}.
The simulation is performed on the training split of the MVTec \gls{d2s} dataset \cite{Paddo18-D2S}, the test split is not altered.
Thereby, two variants of the \gls{d2s} dataset are used, in the training split the normal version of has 4380 images and 6900 annotations, while the augmented version contains 14380 images and 83337 annotations.
The goodness of the newly created annotations $ \textbf{Y}_{new} $ is at least at 0.91 \gls{miou}, as presented in Table \ref{tab:ch5:results_annot_usability}.

% Train model.
In the second step, the new annotations $ \textbf{Y}_{new} $ of the train split, are used to train semantic segmentation models $ m $.
As backbone of a model the Mask-RCNN \cite{He17-MaskR-CNN} is used, which is pretrained on the COCO dataset \cite{Lin14-Coco}.
All trainings are performed with the same settings, to enable a fair comparison.
A training lasts $ n_{epochs} = 30 $ epochs, with the learning rate $ \lambda = 0.001 $ and \gls{sgd} optimizer \cite{Ruder16-SGD}.
To establish a benchmark for comparison, also two models are trained on the original \gls{gt} of the normal and augmented \gls{d2s} dataset.
The models itself do not contain any novelty, but are just used to be trained with the new annotations $ \textbf{Y}_{new} $.

% Eval model.
Third, the performance of the created model is evaluated by the \gls{miou} and the \gls{ap} at different levels if \gls{iou}, as presented in Table \ref{tab:ch5:results_annot_usability}.
Based on the \gls{map}, the models trained on the \gls{dextr} and \gls{iog} annotations, perform approximately as good as the model trained on the original \gls{gt}.
This approximately applies to both \gls{d2s} dataset, but for the augmented \gls{d2s} dataset the model trained on the original \gls{gt} performs slightly better.

%For the \gls{d2s} dataset the model trained on the \gls{dextr} annotations even achieves a \gls{map} $ mAP = 0.4459 $ and therefore is slightly better than the model trained on the original \gls{gt} with $ mAP = 0.4398 $.
However, differences between the models can be observed when considering the IoU for certain degrees of IoU.
The \gls{ap} for annotations at $ IoU < 0.75 $ are at an equal level for all methods.
From there on, the \gls{ap} tends to decrease for a higher \gls{iou}.
This observation is extreme for $ IoU > 0.95 $. 
Exemplary for the models trained on the normal data set, the original \gls{gt} enables a performance of $ AP = 0.2198 $, while the performance of the \gls{dextr} and \gls{iog} model dropped to $ AP = 0.1549 $ respectively $ AP = 0.0384 $.
This effect also occurs for the augmented \gls{d2s} dataset.
From this follows that, the models trained on the \gls{dextr} and \gls{iog} annotations are not able to make prediction with a high level of detail.
The missing details in the model predictions, is probably caused by the \gls{dextr} and \gls{iog} annotations, which are also not perfect at the detail level.
So, in order to make create predictions with very high \gls{iou} and rich detail, the original \gls{gt} delivers better results.
%TODO reread the upper part once.


\begin{table}[h!]
	\centering
	\begin{tabular}{ ll|c c c|c c c}
		\toprule
								& Dataset				& \multicolumn{3}{c}{D2S} 			& \multicolumn{3}{c}{D2S augmented} \\		 
								& 						& GT		& DEXTR		& IOG		& GT		& DEXTR		& IOG		\\
		\midrule
		$ \textbf{Y}_{new} $ 	& mIoU 					& 1.0		& 0.9554	& 0.9271	& 1.0		& 0.9102	& 0.9209	\\
		\midrule
		$ m $					& mAP 					& 0.4398	& 0.4459	& 0.4138	& 0.7581	& 0.7374	& 0.6993	\\
		
								& AP @ $ IoU >= 0.5 $	& 0.4987	& 0.5221	& 0.5179	& 0.8574	& 0.8572	& 0.8543	\\
								& AP @ $ IoU = 0.55 $ 	& 0.4926	& 0.5145	& 0.5135	& 0.8565	& 0.8518	& 0.8502	\\
								& AP @ $ IoU = 0.6 $	& 0.4859	& 0.5063	& 0.5083	& 0.8526	& 0.8471	& 0.8451	\\
								& AP @ $ IoU = 0.65 $ 	& 0.4834	& 0.5035	& 0.5038	& 0.8476	& 0.8419	& 0.8343	\\
								& AP @ $ IoU = 0.7 $	& 0.4788	& 0.4967	& 0.4955	& 0.8425	& 0.8336	& 0.8162	\\
								& AP @ $ IoU = 0.75 $ 	& 0.4685	& 0.4876	& 0.4715	& 0.8276	& 0.8188	& 0.7899	\\
								& AP @ $ IoU = 0.8 $	& 0.4540	& 0.4687	& 0.4381	& 0.8016	& 0.7849	& 0.7370	\\
								& AP @ $ IoU = 0.85 $ 	& 0.4293	& 0.4401	& 0.3850	& 0.7641	& 0.7328	& 0.6718	\\
								& AP @ $ IoU = 0.9 $	& 0.3868	& 0.3651	& 0.2662	& 0.6617	& 0.6008	& 0.5050	\\
								& AP @ $ IoU = 0.95 $ 	& 0.2198	& 0.1549	& 0.0384	& 0.2694	& 0.2047	& 0.0982	\\
		\bottomrule
	\end{tabular}
	\caption[Performance of models trained with DEXTR and IOG annotations]{
		Performance of the \gls{dextr} and \gls{iog} annotations and the trained models.
		This setup was performed on the \gls{d2s} and the augmented \gls{d2s} dataset.
		The performance of the annotations created by the \gls{dextr} and \gls{iog} model is measured by the \gls{miou}.
		For each dataset three models were trained based on the original \gls{gt}, the \gls{dextr}, and \gls{iog} annotations.
		The performance of the model is evaluated with the \gls{map} and the \gls{ap} for different \gls{iou} levels.		
	}\label{tab:ch5:results_annot_usability}
\end{table}
%TODO determine if $ IoU >= 0.5 $ ot $ IoU = 0.5 $

In conclusion, the statement from Manisis \etal, that 
Finally, the statement of Manisis \etal is reviewed again, which claims that models trained on \gls{dextr} annotations perform equally well \cite{Man18-DEXTR}.
It can be said, that the \gls{dextr} and \gls{iog} annotations are suitable to train a new model, which performs equivalent on a lower $ IoU$ level and the \gls{map}.
However, for a high \gls{iou} the model trained on the \gls{dextr} annotations performs a bit worse, while the one from \gls{iog} annotations performs significantly worse.
Depending on the required level of precision of the model, the \gls{dextr} and \gls{iog} annotations are suitable for training.

It has to be noted, that these experiments are only performed on one dataset and the underlying model was already pretrained.
So, no general evidence is provided, but the basic capabilities have been demonstrated.
% finetuning
This approach may be especially useful, in the field of \textit{transfer learning}, to finetune a pretrained model with new annotations created by interactive segmentation methods.

