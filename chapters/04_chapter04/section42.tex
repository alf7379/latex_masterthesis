% !TeX root = ../../main.tex
% Add the above to each chapter to make compiling the PDF easier in some editors.

\section{Method Selection}\label{ord:ch4:sec2}

The benchmark study includes four methods to interactively create segmentation masks, as introduced in Chapter \ref{ord:ch3}. 
In the following their selection is reasoned.

Polygon drawing does not contain any smart algorithms, but just processes the user input.
This method is included, in order to establish a baseline, that realistically represents manual labeling.
Watershed transformation is included to represent the segmentation algorithms based on classical image processing.
With \gls{iog} and \gls{dextr} two \gls{dl} methods are part of the benchmark study, that represent the current state-of-the-art approaches.
Based on the comparison from Table \ref{tab:ch2:interactive-stae-of-the-art}, the \gls{iog} method performs best, by processing user clicks on foreground and background.
While the \gls{dextr} method only processed user clicks on the boundary of the object.

In order to keep the scope of this thesis feasible, the benchmark study contains these four interactive segmentation methods.

% State of the art
\begin{table}[h!]
	\centering
	\begin{tabular}{l|c c c c}
		Method					& polygon 	& watershed & DEXTR & IOG 	\\
		\hline
		Number of annotations 	& 764 		&  	785		& 806 	& 643 	\\
	\end{tabular}
	\caption[Overview of benchmark methods]{
		Distribution of the benchmark annotations over the four benchmark methods.
		An image may contain multiple objects to annotate, that belong to the same image domain.
	} \label{tab:ch4:methods_overview}
\end{table}
