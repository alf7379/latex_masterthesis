% !TeX root = ../../main.tex
% Add the above to each chapter to make compiling the PDF easier in some editors.

\section{Benchmark Statistics}\label{ord:ch4:sec3}

In the scope of the benchmark study several statistics were recorded and measured, that may be grouped in three parts: segmentation result, user interactions and, user survey.

% #1 Performance IoU
First, the saved segmentation results are used to calculate the \gls{iou} as measure of accuracy for each user created annotation.

% #2 User interactions.
Second, the user interactions were recorded, that are required to perform the interactive methods.
% Based on these statistics the usability of the labeling process is evaluated.
These statistics on the user interactions exist for every annotation, the most important are presented in the following:
\begin{itemize}
	\item \textbf{Used method}, the interactive method used to create the annotation.
	\item \textbf{Annotation time}, the time spent to create the annotation.
	\item \textbf{Number of clicks}, the number of clicks used.
	\item \textbf{Number of strokes}, the amount of strokes drawn for the annotation.
	\item \textbf{Stroke time}, the time spent on each stroke, in order to differentiate between short and long strokes.
	\item \textbf{Number of foreground clicks}, the amount of clicks on the foreground.
	\item \textbf{Number of background clicks}, the amount of clicks on the background.
	
	\begin{comment}
	% image_time
	% image_id
	% single_methods
	% single_box_times
	% single_mask_times
	% single_hole_time
	% single_clicks
	% single_strokes
	% single_stroke_times
	% single_change_box_time
	% single_change_mask_time
	% single_change_clicks
	% single_change_strokes
	% single_change_strokes_times
	% fg_click_row
	% fg_click_col
	% num_fg_clicks
	% idx_fg_clicks
	% bg_click_row
	% bg_click_col
	% num_bg_clicks
	% idx_bg_clicks
	% bbox_click_row1
	% bbox_click_col1
	% bbox_click_row2
	% bbox_click_col2
	% idx_bbox_clicks
	% fg_click_region
	% bg_click_region
	\end{comment}	
\end{itemize}

For illustration an exemplary part of the recorded statistics is presented in Table \ref{tab:ch4:presentation_statistics}.

% #3 Survey to get an idea about the soft factors.
Third, the participants filled out a survey, after completing the benchmark study.
This survey enables insights on the usability and user experience of the individual methods.


\begin{table}[h!]
	\centering
	\resizebox{\textwidth}{!}{
	\begin{tabular}{ c c c c c c c c}
		\toprule	 
		Method 		& $ IoU $ 	& $ time_{annot} $	& $ n_{clicks} $ 	& $ n_{strokes} $ 	& $ time_{strokes} $ 	& Domain		& User ID \\
		\midrule		
		polygon 	& 0.9819 	& 51.99				& 35				& 0					& 0						& Urban 		& User 13 \\
		polygon 	& 0.9156 	& 41.07				& 8					& 4					& 4.47					& Industrial 	& User 9  \\
		watershed	& 0.7134 	& 23.08				& 13				& 1					& 1.29					& Industrial  	& User 2  \\
		watershed	& 0.3346 	& 12.3				& 2					& 4					& 4.82					& Anomaly	  	& User 23 \\
		DEXTR		& 0.8897 	& 7.54				& 4					& 0					& 0						& Standard		& User 29 \\
		DEXTR		& 0.8688 	& 24.85				& 13				& 0					& 0						& Industrial	& User 23 \\
		IOG			& 0.8944 	& 17.0				& 5					& 1					& 1.9					& Industrial	& User 27 \\
		IOG			& 0.7269 	& 28.44				& 6					& 1					& 5.09					& Standard		& User 19 \\
		\bottomrule
	\end{tabular}}
	\caption[Sample presentation of recorded user statistics]{
		Sample presentation of recorded statistics, to illustrate what kind of data is recorded and worked with.
	}\label{tab:ch4:presentation_statistics}
\end{table}
% ClassName, Attributes - missing ID