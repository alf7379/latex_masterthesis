% !TeX root = ../../main.tex
% Add the above to each chapter to make compiling the PDF easier in some editors.

\section{Benchmark Statistics}\label{ord:ch4:sec3}

In the scope of the benchmark study several statistics were recorded and measured, that may be grouped in three parts: segmentation result, user interactions and, user survey.

% #1 Performance IoU
First, the saved segmentation results are used to calculate the \gls{iou} as measure of performance for each user created annotation.

% #2 User interactions.
Second, the user interactions were recorded, that are required to perform the interactive methods.
% TODO what about usability??
% Based on these statistics the usability of the labeling process is evaluated.
These statistics on the user interactions exist for every annotation, the most important are presented in the following:
\begin{itemize}
	\item \textbf{Used method}, the interactive method used to create the annotation.
	\item \textbf{Annotation time}, the time spent to create the annotation.
	\item \textbf{Number of clicks}, the number of clicks used.
	\item \textbf{Number of strokes}, the amount of strokes drawn for the annotation.
	\item \textbf{Stroke time}, the time spent on each stroke, in order to differentiate between short and long strokes.
	\item \textbf{Number foreground clicks}, the amount of clicks on the foreground.
	\item \textbf{Number background clicks}, the amount of clicks on the background.
	% image_time
	% image_id
	% single_methods
	% single_box_times
	% single_mask_times
	% single_hole_time
	% single_clicks
	% single_strokes
	% single_stroke_times
	% single_change_box_time
	% single_change_mask_time
	% single_change_clicks
	% single_change_strokes
	% single_change_strokes_times
	% fg_click_row
	% fg_click_col
	% num_fg_clicks
	% idx_fg_clicks
	% bg_click_row
	% bg_click_col
	% num_bg_clicks
	% idx_bg_clicks
	% bbox_click_row1
	% bbox_click_col1
	% bbox_click_row2
	% bbox_click_col2
	% idx_bbox_clicks
	% fg_click_region
	% bg_click_region
	
\end{itemize}

% #3 Survey to get an idea about the soft factors.
Third, the participants filled out a survey, after completing the benchmark study.
This survey enables an insight on the soft factors, that are not measurable.

% TODO table taht provides insights on the pandas datarfame