% !TeX root = ../../main.tex
% Add the above to each chapter to make compiling the PDF easier in some editors.

\section{Polygon Drawing}\label{ord:ch3:sec1}

In contrast to the previously introduced interactive segmentation approaches, this manual polygon drawing acts as a baseline.
By manually setting points on the border of the object in the image a polygon is created.
From three points on a polygon is spanned, where each point functions as a node.
% Each point has two edges.
The area inside the polygon represents the object mask.

There are two ways to set points.
First, to set one point after another by one mouse click for each point.
Second, to draw multiple points by moving the mouse, this also referred to as \textit{stroking}.
This is done by pressing the left mouse button, moving the mouse and releasing the left mouse button.
On the drawn way multiple points with a certain spacing are set.
% TODO Man könnte hier auch den Threshold-Parameter einführen den man dann kurz beschreibt und wo auch irgendwo steht wie er gewählt wurde.

Further, there are additional features implemented, in order to facilitate the drawing process.
Each new set point is automatically included into the nearest edge of the polygon.
Set points can be relocated by a functionality similar to \textit{drag and drop}.
A set point can be removed by dragging it into a neighboring point, such that only one point remains.
There is the possibility to create holes by drawing a hole-polygon into an existing annotation.

This polygon drawing method is included in the benchmark study to be used as a non-interactive baseline in order to better evaluate the other methods