% !TeX root = ../../main.tex
% Add the above to each chapter to make compiling the PDF easier in some editors.


%%% Motivation
% Fast development in computer science -> ML and automation of processes in the industry.
The development in computer science has continued to make great strides in recent years. 
Here two areas are emphasized in particular for this thesis: \gls{ml} and the automation of industrial processes.
The former is driven by an energetic research community, resulting in novel methods and application areas.
% Increasing automation == increasing application of image processing (-> application of DL).
The latter enjoys the benefits of increasing digitization and the developments in digital image processing \cite{Rah19-IoT}.
Especially for the automation of visual applications, image processing is the main component.
% Semantic Segmentation of Images as valueable method with increasing application.
In this field the segmentation of an image is a valuable instrument, in order to identify and localize objects in images.
Recently, segmentation methods based on \gls{ml} emerged, resulting in advanced performance in the form of higher accuracy and richer detail.

% -> Interactive Semantic Segmentation 
Further, it is possible to apply segmentation methods interactively, whereby the object to be segmented can be determined more precisely by a user.
% Use cases for interactive methods
Possible use cases for interactive segmentation methods are segmentation itself and the creation of new annotations \cite{Man18-DEXTR}.

% Importance of strong generalization capabilities. 
In industrial contexts interactive methods may be used for various applications.
Thereby the interactive method can be operated differently by various users.
Also, a wide range of images from unknown domains may be applied. 
To overcome these challenges, interactive methods must have good generalization capabilities over several image domains and user behaviors.

A popular application for such advanced image processing methods is the automation of repetitive processes in industrial manufacturing.
This includes, for example, the detection of errors, the counting of instances or the control of quality \cite{Chen19-AnomalyDetectionManufacturing}.
\\
\newline
%%% Scope of this thesis
This thesis attempts to give an overview of interactive methods for semantic segmentation and their abilities.
First, basic types of interactive segmentation methods are presented.
Next, four methods are explained in detail, two of which represent the state-of-the-art and are based on \gls{ml}.
The core of this work is a benchmark study, where real users apply the presented methods on a custom dataset.
The results from the benchmark are used to examine if interactive segmentation methods based on \gls{ml} improve the process of labeling based on the annotation time and accuracy.
A focus is placed on the generalization capabilities of the methods, which are fundamental for a successful application.
Last, Chapter \ref{ord:ch6} summarizes the results in a conclusion.

% TODO inlcude research question here??

