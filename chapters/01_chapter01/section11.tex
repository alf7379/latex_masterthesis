% !TeX root = ../../main.tex
% Add the above to each chapter to make compiling the PDF easier in some editors.


%%% Motivation
% Fast development in computer science -> ML and automation of processes in the industry.
The development in computer science has continued to make great strides in recent years. 
For this thesis two areas are emphasized in particular: \gls{ml} and the automation of industrial processes.
The former is driven by an energetic research community, resulting in novel methods and application areas.
% Increasing automation == increasing application of image processing (-> application of DL).
The latter enjoys the benefits of increasing digitization and the developments in image processing due to \gls{ml} .
% Especially, for the automation of visual applications, the main component, image processing, has evolved revolutionary.
% Semantic Segmentation of Images as valueable method with increasing application.
A widely used method in image processing is segmentation, which is a valuable instrument, in order to identify and localize objects in images.
Recently, segmentation methods based on \gls{ml} emerged, resulting in advanced performance in the form of higher accuracy and richer detail.
A popular application is the automation of repetitive processes in industrial manufacturing as the detection of errors, the counting of instances or the control of quality \cite{Rah19-IoT} \cite{Chen19-AnomalyDetectionManufacturing}.

% -> Interactive Semantic Segmentation 
Further, it is possible to apply interactive segmentation methods in order to segment objects with the support of a user.
Thereby, the user interaction often is realized by mouse clicks on the object of interest.
% Use cases for interactive methods
A promising use cases for interactive segmentation methods is the creation of annotations, that further may be used as \gls{gt} \cite{Man18-DEXTR}.
%In industrial contexts interactive methods may be used for various applications.
% Importance of strong generalization capabilities. 
For interactive segmentation methods to become established and widely used, they must have strong generalization capabilities.
This includes the generalization over various image domains and users, which use the method differently.
\\
\newline
%%% Scope of this thesis
This thesis gives an overview of interactive methods for semantic segmentation and evaluates the abilities of specific methods in detail.
In Chapter \ref{ord:ch2} basic types of interactive segmentation methods are presented.
Four interactive methods are extensively explained in Chapter \ref{ord:ch3}, two of which represent the state-of-the-art and are based on \gls{ml}.
In Chapter \ref{ord:ch4} a benchmark study is presented, which is the core of this thesis.
Thereby, real users apply the presented interactive segmentation methods to label a custom created dataset.
The results from the benchmark study are evaluated in Chapter \ref{ord:ch5}.
Here it is examined to what extend interactive segmentation methods based on \gls{ml} improve the process of labeling based on the annotation time and accuracy.
A focus is placed on the generalization capabilities of the methods, which are fundamental for a successful application.
It is also investigated whether annotations created by interactive segmentation models are suitable to be used as training data to train a new \gls{ml} model.
Last, a final conclusion over the results of the benchmark study and the performance of the methods is provided in Chapter \ref{ord:ch6}.

