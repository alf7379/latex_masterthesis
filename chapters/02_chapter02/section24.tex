% !TeX root = ../../main.tex
% Add the above to each chapter to make compiling the PDF easier in some editors.

\section{Statistics}\label{ord:ch2:sec4}

This Section provides the theoretical foundation for the statistical evaluation, which is performed in Chapter \ref{ord:ch5}.
Only the principles and methods are briefly introduced, which are also relevant in the later evaluation, a general introduction into the field of statistics is given in \cite{PS16-Statistics} or \cite{Dodge08-Statistics}.
In statistics the data, which is analyzed, is also referred to as sample.


\subsection{Normality Assumption in Statistics} \label{ord:ch2:sec4:subsec1}
%TODO ref normal distribution
A common assumption for statistical analysis is, that the data points of a sample follow a normal distribution.
This is often a requirement for the application of statistical methods, this group of methods is also referred to as parametric methods.
In contrast, non-parametric methods do not require normally distributed samples.
In order to determine if a sample is normally distributed, two types of methods exist: normality tests or visual plots.
Normality tests control if a sample is drawn from a normal population.
When using large real world datasets, these tests may not be suitable, because even small deviations may be detected and interpreted as evidence of non-normality.
As a result, visual plots are frequently used to confirm the distribution of a sample.
To visualize the distribution and compare it to the normal distribution, the data points of the sample may be plotted in a histogram.
Further, normal characteristics can be illustrated by a normal probability plot.
In a normal probability plot the data points are aligned at the diagonal line, if they are is distributed normally. \footnote{Wolfram Research (2010), ProbabilityPlot \url{https://reference.wolfram.com/language/ref/ProbabilityPlot.html}}

If a sample is not distributed normally, \cite{PS16-Statistics} suggests to transform the sample, in order to achieve normality.
This transformation is realized by the application of a mathematical function to every data point in the sample.
Thereby, the application of a function tries to shift the data into a normal distribution.
The effect of the transformation can be controlled by a histogram or probability plot.


\subsection{Student's t-test}\label{ord:ch2:sec4:subsec2}

The Student's t-test is a hypothesis test, that belongs to the group of parametric tests and therefore expects normal distributed samples.
The details of hypothesis test in the form of ttests are illustrated in \cite{Dodge08-Statistics} \cite{FisherBox81-StudentT} \cite{RK06-HypothesisTesting}.

In general, a hypothesis test is a statistical method to investigate an assumption on a specific variable in a sample.
The variable is divided in $N$ groups, which are also referred to as factors.
An assumption is made whether the different groups differ or not.
The assumption is represented by a so-called null hypothesis $H_{0}$
\begin{equation}
	H_{0}: \theta_{1} = \theta_{2}
\end{equation}
with $N=2$ factors and $\theta$ representing the factor as \eg mean or median value.
$H_{0}$ mostly states, that there is not difference between the factors of the sample.
The counterpart of $H_{0}$ is the alternative hypothesis $H_{A}$,
%\begin{equation}
%	H_{A}: \theta_{1} > \theta_{2} or \theta_{1} \not= \theta_{2} or \theta_{1} < \theta_{2}
%\end{equation} 
which contrasts with $H_{0}$ and usually assumes a difference between the factors.
The evaluation of $H_{0}$ takes place in the measure of the significance level $\alpha$, which represents the probability of a $H_{0}$ being wrongly rejected ($\textnormal{\textit{false negative}}$) or wrongly accepted ($\textnormal{\textit{false positive}}$).
Based on the properties of the sample (\eg mean, variance, standard deviation) a statistic is used to calculate the $\textnormal{\textit{p-value}}$.
If the $\textnormal{\textit{p-value}}$ is smaller than the chosen $\alpha$, $H_{0}$ is rejected, while $H_{A}$ is accepted and vice versa.
In evaluations hypothesis tests are applied to identify statistical relevant differences between factors and prevent the misinterpretation of differences caused by accident.

The Student's t-test is a widely used hypothesis test, that is used to compare two factors of one variable.
The evaluation is based on the mean value $\overline{x}$ of each factor.
% Hartmann, K., Krois, J., Waske, B. (2018): E-Learning Project SOGA: Statistics and Geospatial Data Analysis. Department of Earth Sciences, Freie Universitaet Berlin.


\subsection{Mann-Whitney U-Test}\label{ord:ch2:sec4:subsec3}

The Mann-Whitney U-Test \cite{MW47-MannWhitneyTest} is a non-parametric hypothesis test.
This test is also referred to as non-parametric counterpart to the Student's t-test.
Mann-Whitney compares independent random samples of two factors.
These random samples are ranked based on their value.
Therefore the tendency $ten \left( x \right) $ in the sample is determined.
$H_{0}$ assumes that both samples have the same tendency $ten \left( x \right) $, while $H_{A}$ states the opposite.


\subsection{Kruskal-Wallis Test}\label{ord:ch2:sec4:subsec4}
The Kruskal-Wallis test \cite{KW53-KruskalWallisTest} is a non-parametric hypothesis test.
Therefore, samples are not required to be distributed normally.
Kruskal-Wallis enables the comparison of two or more factors from a variable.
Kruskal-Wallis is also known as the non-parametric alternative to \gls{anova}, which is a popular parametric test to evaluate multiple factors.
The Kruskal-Wallis test compares the median values $med \left( x \right) $ of each factor.
As usual for hypothesis tests, in $H_{0}$ it is assumed, that the median values of all factors are equal.
$H_{A}$ states, that there is a significant difference between at least one median value.
The number of samples in each factor must be equal in order to perform this test.
However, the Kruskal-Wallis test does not state which of the factors differs significantly.
To identify the deviating factor a post analysis is performed by \eg Connover test \cite{CI79-ConnocerTest}, Dunn test \cite{Dunn64-DunnTest}, or DSCF test \cite{CF91-dscf}. \footnote{Scikit-Posthocs (2021), posthocs API \url{https://scikit-posthocs.readthedocs.io/en/latest/posthocs_api/}} 

