% !TeX root = ../../main.tex
% Add the above to each chapter to make compiling the PDF easier in some editors.

\section{Interactive Semantic Segmentation}\label{ord:ch2:sec3}

While semantic segmentation performs the task of segmenting an image just with the image itself, Interactive Semantic Segmentation takes advantage of additional information interactively provided by an user.
The idea of this concept is to enhance the segmentation result by adding a new sort of information, that is already processed by an user.
Because of this, the user input has great value for the network and provides high level guidance for the task of segmentation.
Depending on the type of interaction, the receipt of the user input may be more or less elaborately, which leads to a weighing of the advantages and disadvantages.
On the one side interactively provided user input contains high level information, but on the other side user interactions may be very expensive, especially in the context of the big amount of images in datasets required for deep learning tasks.
%TODO declare if these methods are used for semantic segmentation or instance segmentation.
In the following basic concepts of interactive semantic segmentation are introduced by presenting specific methods.
 

\subsection{Subsection}\label{ord:ch2:sec3:subsec1}

\subsection{Points from users}\label{ord:ch2:sec3:subsec2}

A common and well known practice to obtain user input dates back to 1985, when Apple with co-founder Steve Jobs introduced the \textbf{Xerox}-computer with the first stage of development of the computer mouse.

Multiple methods use user-clicks on specific characteristics o
