% !TeX root = ../../main.tex
% Add the above to each chapter to make compiling the PDF easier in some editors.

\section{ML, DL, CNNs}\label{ord:ch2:sec1}

The last decade was revolutionary for the sector of information technology.
Due to technical advancement, the computational power of processors, especially Graphical Processing Units (GPU) rise significantly.
Further, the wider creation and use of data introduced the domain of big data, that allows operators to gain more insights and benefits.
Both of these recent advancements benefited another field of study commonly described as Artificial Intelligence (AI).
The term AI describes machines that are show characteristics of human intelligence that allow them to handle various tasks.
But there are several gradations, that hide behind the powerful bus word AI and are illustrated in this section.

\subsection{Machine Learning}\label{ord:ch2:sec1:subsec1}

Machine Learning (ML) is a subsection of AI and is based on the ability of algorithms to analyze, detect and learn patterns in various kinds of data. 
Applying these learned pattern ML algorithms may reach human level performance or even better, but they are limited specifically to their scope.
For other tasks mostly a new ML algorithms needs to be defined \cite{HR18-AI}.
In contrast AI may be able to solve various tasks and learn independently by showing strong characteristics of human intelligence.

In ML


\subsection{Deep Learning}\label{ord:ch2:sec1:subsec2}
\subsection{Convolutional Neural Networks}\label{ord:ch2:sec1:subsec3}

% See~\autoref{tab:ch2:sample}, \autoref{fig:ch2:sample-drawing}, \autoref{fig:ch2:sample-plot}, \autoref{fig:ch2:sample-listing}.

\begin{table}[htpb]
  \caption[Example table]{An example for a simple table.}\label{tab:ch2:sample}
  \centering
  \begin{tabular}{l l l l}
    \toprule
      A & B & C & D \\
    \midrule
      1 & 2 & 1 & 2 \\
      2 & 3 & 2 & 3 \\
    \bottomrule
  \end{tabular}
\end{table}

\begin{figure}[htpb]
  \centering
  % This should probably go into a file in figures/
  \begin{tikzpicture}[node distance=3cm]
    \node (R0) {$R_1$};
    \node (R1) [right of=R0] {$R_2$};
    \node (R2) [below of=R1] {$R_4$};
    \node (R3) [below of=R0] {$R_3$};
    \node (R4) [right of=R1] {$R_5$};

    \path[every node]
      (R0) edge (R1)
      (R0) edge (R3)
      (R3) edge (R2)
      (R2) edge (R1)
      (R1) edge (R4);
  \end{tikzpicture}
  \caption[Example drawing]{An example for a simple drawing.}\label{fig:ch2:sample-drawing}
\end{figure}

\begin{figure}[htpb]
  \centering

  \pgfplotstableset{col sep=&, row sep=\\}
  % This should probably go into a file in data/
  \pgfplotstableread{
    a & b    \\
    1 & 1000 \\
    2 & 1500 \\
    3 & 1600 \\
  }\exampleA
  \pgfplotstableread{
    a & b    \\
    1 & 1200 \\
    2 & 800 \\
    3 & 1400 \\
  }\exampleB
  % This should probably go into a file in figures/
  \begin{tikzpicture}
    \begin{axis}[
        ymin=0,
        legend style={legend pos=south east},
        grid,
        thick,
        ylabel=Y,
        xlabel=X
      ]
      \addplot table[x=a, y=b]{\exampleA};
      \addlegendentry{Example A};
      \addplot table[x=a, y=b]{\exampleB};
      \addlegendentry{Example B};
    \end{axis}
  \end{tikzpicture}
  \caption[Example plot]{An example for a simple plot.}\label{fig:ch1:sample-plot}
\end{figure}

\begin{figure}[htpb]
  \centering
  \begin{tabular}{c}
  \begin{lstlisting}[language=SQL]
    SELECT * FROM tbl WHERE tbl.str = "str"
  \end{lstlisting}
  \end{tabular}
  \caption[Example listing]{An example for a source code listing.}\label{fig:ch2:sample-listing}
\end{figure}
