% !TeX root = ../../main.tex
% Add the above to each chapter to make compiling the PDF easier in some editors.

\section{ML, DL and CNN}\label{ord:ch2:sec1}

The last decade was revolutionary for the sector of information technology.
Due to technical advancement, the computational power of processors, especially \gls{gpu} rise significantly.
Further, the wider creation and use of data introduced the domain of big data, that allows operators to gain more insights and benefits.
Both of these recent advancements benefited another field of study commonly described as \gls{ai}.
The term \gls{ai} describes machines that are show characteristics of human intelligence that allow them to handle various tasks.
But there are several gradations, that hide behind the powerful bus word \gls{ai} and are illustrated in this section.

\subsection{Machine Learning}\label{ord:ch2:sec1:subsec1}

\gls{ml} is a subsection of \gls{ai} and is based on the ability of algorithms to analyze, detect and learn patterns in various kinds of data. 
Applying these learned pattern \gls{ml} algorithms may reach human level performance or even better, but they are limited specifically to their scope.
For other tasks mostly a new \gls{ml} algorithms needs to be defined 
\footnote{Henning Steiner, Hessischer Rundfunk: \textit{Selbstlernende Maschinen - wie Künstliche Intelligenz entsteht}: \url{https://www.hr-inforadio.de/podcast/wissen/selbstlernende-maschinen---wie-kuenstliche-intelligenz-entsteht,podcast-episode-53312.html}}.
In contrast \gls{ai} may be able to solve various tasks and learn independently by showing strong characteristics of human intelligence.



\subsection{Deep Learning}\label{ord:ch2:sec1:subsec2}
\subsection{Convolutional Neural Networks}\label{ord:ch2:sec1:subsec3}
